\section{Osculating Circles Estimator}

Osculating Circles Estimator is an algorithm that calculate the curvature of a contour, by pointing out the tangent circle at each point of the contour. 

\paragraph{}
For implementing this part, we used the definition of digital straight segment and digital straight line. 

\subsection{Digital Straight Line}
Digital Straight Line (DSL) is defined by 4 values : $D(a, b, \mu, \omega)$, with $a, b, c, d \in \mathbb{Z}$ and $gcd(a, b) = 1$. $a/b$ is called a sloped of $D$, $\mu$ is an intercept and $\omega$ is the thickness of $D$.

\paragraph{}
Every points that belong to $D(a, b, \mu, \omega)$ must satisfy :
$$
\mu \leq ax-by < \mu + \omega
$$

In DGtal, there are 2 types of DSL :
\begin{itemize}
\item Naive Digital Straight Line
\item Standard Digital Straight Line
\end{itemize}

These type of DSL is made by specifying the value \textit{thickness}. 

\begin{itemize}
\item Naive DSL : $\omega = max(|a|, |b|)$
\item Standard DSL : $\omega = |a| + |b|$
\end{itemize}

\subsection{Digital Straight Segment}
Digital Straight Segment (DSS) is the set of points that belong to a digital straight line. Just like DSL, in DGtal there are also two specifics DSS : Naive DSS and Standard DSS.

\subsection{Implementation}
In this project, we used the standard DSL and DSS to implement this algorithm. 