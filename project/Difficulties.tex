\section{Difficulties}

Thorough this project, we have encountered many difficulties, but in contrast, we have obtained many knowledges. First of all is the compilation process of this project. We have to use \textbf{CMAKE} system, which is new to us. 

\paragraph{}
Secondly, also in the compilation project, is linking the library and the project by \textbf{CMAKE}. There are two kinds of library that we could classify : one that supports \textbf{CMAKE} and the other don't. But then, once we found the solution for this problem, we could see that \textbf{CMAKE} is a really powerful tool. 

\paragraph{}
Thirdly, the \textbf{DGtal} library. This is the first time that we use a big project as \textbf{DGtal} as a tool in our project. And what trouble us the most, is  the enormous usage of template and generic items. But then, this could be a reference for us, to work in a big project in the future. 

\paragraph{}
Finally, the paralleling algorithm. There were a long time that we have to decide  which should be used : \textbf{CUDA} or \textbf{OpenCL}. \textbf{CUDA} is in fact, easier to write than \textbf{OpenCL}, as \textbf{NVIDIA} provides a lot of template and library, that make the programing easier a lot more than \textbf{OpenCL}. But because of the installation on one of our machines is unsuccessful, we chose \textbf{OpenCL}. And furthermore, \textbf{OpenCL} is available on MAC OSX, and it isn't a \textbf{CMAKE} supported library.