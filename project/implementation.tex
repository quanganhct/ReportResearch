\section{MACHINE'S IMPLEMENTATION}

For examining an array, we need to run through all of the members of the array. So I introduced a variable \textit{indice} which hold the current position of the last array's element that was visited.

$$ indice \in 1..size $$

\paragraph{}
The machine will stop when it find an element that doesn't satisfy the condition in section 3. That means all the member before the \textit{indice} must satisfy the condition.

$$ \forall i \cdot i \in 1..size \land i < indice \implies array(i) \leq array(i+1) $$

\paragraph{}
At first \textit{indice} start with value of 1. Then, as long as it satisfies the condition below, we increase \textit{indice} by 1
$$ indice < size \land array(indice) \leq array(indice+1) $$

\paragraph{}
Now, our array is sorted in ascending order if $ indice \geq size $, and it isn't if $ indice < size \land array(indice) > array(indice+1) $

\paragraph{}
To test that our machine will not stop half way, we need to justify that the expression below is true
$$ (indice \geq size) \lor (indice < size \land array(indice) > array(indice+1)) $$
$$ \lor (indice < size \land array(indice) \leq array(indice+1)) $$ 

\paragraph{}
To test that our machine will not fall into an infinite loop, we need to point out a value that would decrease each time we go through a loop, and is greater than zero. I choose this value : $size - indice$. This value is indeed decreased each time we go through the loop, as \textit{indice} would increase and \textit{size} is fixed, and because of the definition of \textit{indice}, $size - indice \geq 0$